\documentclass[10pt,aspectratio=43]{beamer} 
\usepackage{hyperref}
\usepackage{xeCJK} %导入中文包

\setCJKsansfont{SimHei} 
\setsansfont{Microsoft YaHei}
\setmainfont{Microsoft YaHei}
\usefonttheme[onlymath]{serif}
% other packages
\usepackage{latexsym,amsmath,xcolor,multicol,booktabs,calligra}
\usepackage{graphicx,pstricks,listings,stackengine}
\usepackage[linesnumbered,vlined]{algorithm2e}
%\usepackage[linesnumbered,ruled,vlined]{algorithm2e}
\usepackage{float}    
\author{雷尚远}
\title{定向覆盖模糊测试工具的设计与实现}
\subtitle{毕业设计中期检查}
\institute{南京邮电大学计算机学院}
\date{2023年4月17日}
\usepackage{NJUPT}

% defs
\def\cmd#1{\texttt{\color{red}\footnotesize $\backslash$#1}}
\def\env#1{\texttt{\color{blue}\footnotesize #1}}
\definecolor{deepblue}{rgb}{0,0,0.5}
\definecolor{deepred}{rgb}{0.6,0,0}
\definecolor{deepgreen}{rgb}{0,0.5,0}
\definecolor{halfgray}{gray}{0.55}

\lstset{
    basicstyle=\ttfamily\small,
    keywordstyle=\bfseries\color{deepblue},
    emphstyle=\ttfamily\color{deepred},    % Custom highlighting style
    stringstyle=\color{deepgreen},
    numbers=left,
    numberstyle=\small\color{halfgray},
    rulesepcolor=\color{red!20!green!20!blue!20},
    frame=shadowbox,
}


\begin{document}

\begin{frame}
    \titlepage
    \begin{figure}[htpb]
        \begin{center}
            \includegraphics[width=0.2\linewidth]{pic/NJUPT_Logo.pdf}
        \end{center}
    \end{figure}
\end{frame}

\begin{frame}
    \tableofcontents[sectionstyle=show,subsectionstyle=show/shaded/hide,subsubsectionstyle=show/shaded/hide]
\end{frame}


\section{Background}
\subsection{Pre-Knowledge}
\begin{frame}[fragile] {What Fuzzing is?}
    \only<1>{
        \structure {Defination}
        \begin{itemize} 
            \item \textbf{Fuzzing} Fuzzing is the execution of the PUT using input(s) sampled from an input space (the “fuzz input space”) 
                that protrudes the expected input space of the PUT\cite{manes2019art}.
                \\ - PUT: Program Under Test
            \item \textbf{Fuzz testing} Fuzz testing is the use of fuzzing to test if a PUT violates a correctness policy\cite{manes2019art}.
            \item \textbf{Fuzzer} A fuzzer is a program that performs fuzz testing on a PUT\cite{manes2019art}.
            \item \textbf{Bug Oracle} A bug oracle is a program, perhaps as part of a fuzzer, that determines whether a given execution 
            of the PUT violates a specific correctness policy\cite{manes2019art}.
            \item \textbf{Fuzz Configuration}  A fuzz configuration of a fuzz algorithm comprises the parameter value(s) that control(s) the fuzz algorithm\cite{manes2019art}.
        \end{itemize}
    } 
    \only<2>{
            \begin{block}{Fuzz Testing}
            \begin{algorithm}[H]
                \small
                \DontPrintSemicolon
                \SetKwSty{algokeywordsty}
                \SetFuncSty{algofuncsty}
                \SetDataSty{algodatasty}
                \SetArgSty{algoargsty}
                \SetCommentSty{algocmtsty}
                \SetKw{break}{break}
                \SetKw{not}{not}
                \SetKwFunction{preprocess}{\textsc{Preprocess}}
                \SetKwFunction{schedule}{\textsc{Schedule}}
                \SetKwFunction{inputGen}{\textsc{InputGen}}
                \SetKwFunction{inputEval}{\textsc{InputEval}}
                \SetKwFunction{confUpdate}{\textsc{ConfUpdate}}
                \SetKwFunction{continue}{\textsc{Continue}}
                \SetKwFunction{isBug}{isBug}
                \SetKwFunction{getProgram}{getProgram}
                \SetKwData{newbugs}{$\bugs^\prime$}
                \KwIn{\confs, \timeout}
                \KwOut{\bugs \tcp{a finite set of bugs}}
                $\bugs \gets \varnothing$\;
                $\confs \gets \preprocess{$\confs$}$\;
                \While {$\currtime < \timeout \land \continue{\confs}$}{
                  \conf $\gets \schedule{\confs, \currtime, \timeout}$\;
                  \testcases $\gets \inputGen{\conf}$\;
                  \tcp{\bugoracle is embedded in a fuzzer}
                  \newbugs, \execinfos $\gets$ \inputEval{\conf, \testcases, \bugoracle}\;
                  $\confs \gets \confUpdate{\confs, \conf, \execinfos}$\;
                  $\bugs \gets \bugs \cup \newbugs$\;
                }
                \Return{\bugs}\;
                %\caption[short]{fuzz testing}
            \end{algorithm} 
        \end{block}
    } 
    \only<3>{
        \begin{block}{Fuzz Testing}
        \begin{minipage}[t]{0.65\linewidth}
        \begin{algorithm}[H]
            \footnotesize
            \DontPrintSemicolon
            \SetKwSty{algokeywordsty}
            \SetFuncSty{algofuncsty}
            \SetDataSty{algodatasty}
            \SetArgSty{algoargsty}
            \SetCommentSty{algocmtsty}
            \SetKw{break}{break}
            \SetKw{not}{not}
            \SetKwFunction{preprocess}{\textsc{Preprocess}}
            \SetKwFunction{schedule}{\textsc{Schedule}}
            \SetKwFunction{inputGen}{\textsc{InputGen}}
            \SetKwFunction{inputEval}{\textsc{InputEval}}
            \SetKwFunction{confUpdate}{\textsc{ConfUpdate}}
            \SetKwFunction{continue}{\textsc{Continue}}
            \SetKwFunction{isBug}{isBug}
            \SetKwFunction{getProgram}{getProgram}
            \SetKwData{newbugs}{$\bugs^\prime$}
            \HiLi \KwIn{\confs, \timeout}
            \HiLi \KwOut{\bugs \tcp{a finite set of bugs}}
            $\bugs \gets \varnothing$\;
            $\confs \gets \preprocess{$\confs$}$\;
            \While {$\currtime < \timeout \land \continue{\confs}$}{
              \conf $\gets \schedule{\confs, \currtime, \timeout}$\;
              \testcases $\gets \inputGen{\conf}$\;
              \tcp{\bugoracle is embedded in a fuzzer}
              \newbugs, \execinfos $\gets$ \inputEval{\conf, \testcases, \bugoracle}\;
              $\confs \gets \confUpdate{\confs, \conf, \execinfos}$\;
              $\bugs \gets \bugs \cup \newbugs$\;
            }
            \Return{\bugs}\;
        \end{algorithm} 
        \end{minipage} 
        \begin{minipage}[t]{0.34\linewidth}
            {
                \footnotesize{
                    \begin{itemize}  
                        \item  \confs :a set of fuzz configurations
                        \item  \timeout: timeout
                        \item  \bugs: a set of discovered bugs
                    \end{itemize} 
                }
            }
        \end{minipage}
        \end{block}
    }  

    \only<4>{
        \begin{block}{Fuzz Testing}
        \begin{minipage}[t]{0.65\linewidth}
        \vspace{0pt}
        \centering
        \begin{algorithm}[H]
            \footnotesize
            \DontPrintSemicolon
            \SetKwSty{algokeywordsty}
            \SetFuncSty{algofuncsty}
            \SetDataSty{algodatasty}
            \SetArgSty{algoargsty}
            \SetCommentSty{algocmtsty}
            \SetKw{break}{break}
            \SetKw{not}{not}
            \SetKwFunction{preprocess}{\textsc{Preprocess}}
            \SetKwFunction{schedule}{\textsc{Schedule}}
            \SetKwFunction{inputGen}{\textsc{InputGen}}
            \SetKwFunction{inputEval}{\textsc{InputEval}}
            \SetKwFunction{confUpdate}{\textsc{ConfUpdate}}
            \SetKwFunction{continue}{\textsc{Continue}}
            \SetKwFunction{isBug}{isBug}
            \SetKwFunction{getProgram}{getProgram}
            \SetKwData{newbugs}{$\bugs^\prime$}
            \KwIn{\confs, \timeout}
            \KwOut{\bugs \tcp{a finite set of bugs}}
            \HiLi$\bugs \gets \varnothing$\;
            \HiLi$\confs \gets \preprocess{$\confs$}$\;
            \While {$\currtime < \timeout \land \continue{\confs}$}{
              \conf $\gets \schedule{\confs, \currtime, \timeout}$\;
              \testcases $\gets \inputGen{\conf}$\;
              \tcp{\bugoracle is embedded in a fuzzer}
              \newbugs, \execinfos $\gets$ \inputEval{\conf, \testcases, \bugoracle}\;
              $\confs \gets \confUpdate{\confs, \conf, \execinfos}$\;
              $\bugs \gets \bugs \cup \newbugs$\;
            }
            \Return{\bugs}\;
        \end{algorithm} 
        \end{minipage} 
        \begin{minipage}[t]{0.34\linewidth}
            {
                \vspace{0pt}
                \centering
                \preprocess\normalfont(\confs) $\rightarrow \confs$ 
                \scriptsize{
                    \\A user supplies \preprocess with a set of fuzz configurations as input, and it returns a potentially-modified set of fuzz configurations.
                    Depending on the fuzz algorithm, \preprocess may perform a variety of actions such as inserting instrumentation code to PUTs, 
                    or measuring the execution speed of seed files. 
                }
            }
        \end{minipage}
        \end{block}
    }
    



\end{frame}


\subsection{Motivation}
\begin{frame}{Why Grey-box Fuzzing ?}
    \only<1>{
        \structure {classification of fuzzing}
        \begin{itemize} 
            \item  \textbf{Black-box Fuzzing}
                \\ - no program analysis, no feedback
            \item  \textbf{White-box Fuzzing} 
                \\ - mostly program analysis
            \item  \textbf{Grey-box  Fuzzing} 
                \\ - no program analysis, but feedback
        \end{itemize} 
    }

    \only<2>{
        \begin{itemize}  
        \item \structure {Black-box Fuzzing}
          \\ \textbf{Defination:} techniques that do not see the internals of the PUT,and can observe only the input/output behavior of the PUT,
           treating it as a black-box\cite{manes2019art}.
           \\ -No \alert{program analysis}, no \alert{feedback}
        \end{itemize} 
        \begin{picture}(320,250)
            \put(-15,110){\includegraphics[width=8.0cm]{pic/blackbox.pdf}}
        \end{picture} 
    }

    \only<3>{
        \begin{itemize}  
        \item \structure {Black-box Fuzzing}
          \\ \textbf{Defination:} techniques that do not see the internals of the PUT,and can observe only the input/output behavior of the PUT,
           treating it as a black-box\cite{manes2019art}.
           \\ - No \alert{program analysis}, no \alert{feedback}
        \end{itemize} 
        \begin{picture}(320,250)
            \put(-25,110){\includegraphics[width=8.0cm]{pic/blackbox.pdf}}
            \put(190,220){
                \begin{minipage}[t]{0.45\linewidth}
                {
                    \footnotesize{
                        \begin{itemize}  
                            \item  You have no view of the PUT,but have some view of the input/output domain
                            \item  Fuzzing process is not changed according to some feedback 
                            \item  Random mutated (not \alert {effective})
                        \end{itemize} 
                    }
                }
                \end{minipage}
                }
        \end{picture} 
    }
    \only<4>{
        \begin{itemize}  
        \item \structure {White-box Fuzzing}
          \\ \textbf{Defination:} techniques that generates test cases by analyzing the internals of the PUT and the information gathered 
          when executing the PUT\cite{manes2019art}.
          \\ - Requires heavy-weight \alert {program analysis} and constraint solving. 
        \end{itemize} 
        \begin{picture}(320,250)
            \put(-20,120){\includegraphics[width=6.0cm]{pic/cfg.pdf}}
            \put(160,120){\includegraphics[width=6.0cm]{pic/whitebox.pdf}} 
        \end{picture} 
    }
    \only<5>{
        \begin{itemize}  
        \item \structure {White-box Fuzzing}
          \\ \textbf{Defination:} techniques that generates test cases by analyzing the internals of the PUT and the information gathered 
          when executing the PUT\cite{manes2019art}.
          \\ - Requires heavy-weight \alert {program analysis} and constraint solving. 
        \end{itemize}
        \begin{picture}(320,250) 
        \put(-15,110){\includegraphics[width=8.0cm]{pic/whitebox.pdf}}
        \put(200,210){
            \begin{minipage}[t]{0.40\linewidth}
            {
                \footnotesize{
                    \begin{itemize}  
                        \item  You have the view of the PUT state(CFG,CG)
                        \item  Static analysis (effective but not \alert {efficient}!)
                    \end{itemize} 
                }
            }
            \end{minipage}
            }
    \end{picture} 
    }
    \only<6>{
        \begin{itemize}  
        \item \structure {Grey-box Fuzzing}
          \\ \textbf{Defination:} techniques that can obtain \textit{some} information internal to the PUT and/or its executions\cite{manes2019art} to generates test cases.
          \\ - Uses only lightweight instrumentation to glean some program structure
          \\ - And coverage \alert{feedback} 
        \end{itemize} 
        \begin{figure}[htbp]
            \begin{center}
                \includegraphics[width=10.0cm]{pic/greybox.pdf}
            \end{center}
        \end{figure} 
    }
\end{frame}



\subsection{Research Status }
\begin{frame}{ Why Directed Grey-Box Fuzz?}
    \begin{itemize}[<+-| alert@+>] % 当然,除了alert,手动在里面插 \pause 也行
        \item 大家都会\LaTeX{},好多学校都有自己的Beamer主题
        \item 中文支持请选择 Xe\LaTeX{} 编译选项
    \end{itemize}
\end{frame}


\section{研究现状}

\subsection{Beamer主题分类}

\begin{frame}
    \begin{itemize}
        \item 有一些 \LaTeX{} 自带的
        \item 有一些Tsinghua的
        \item 本模板来源自THU Beamer Theme
        \item 但是最初的 \href{http://far.tooold.cn/post/latex/beamertsinghua}{\color{purple}{link}} \cite{origin}已经失效了
        \item 这是原作者在16-17年做的一些ppt:\href{https://github.com/Trinkle23897/oi_slides}{\color{blue}{戳我}}
    \end{itemize}
\end{frame}


\section{研究内容}

\subsection{美化主题}

\begin{frame}{这一份主题与原始的THU Beamer Theme区别在于}
    \begin{itemize}
        \item 顶栏的小点变成一行而不是多行
        \item 中文采用楷书
        \item 修改了主题色为南邮校徽颜色
        \item 参考文献格式按照毕设标准进行了修改
        \item 更多该模板的功能可以参考 \url{https://www.latexstudio.net/archives/4051.html}
        \item 下面列举出了一些Beamer的用法,部分节选自 \url{https://tuna.moe/event/2018/latex/}
    \end{itemize}
\end{frame}

\subsection{如何更好地做Beamer}

\begin{frame}{Why Beamer}
    \begin{itemize}
        \item \LaTeX 广泛用于学术界,期刊会议论文模板
    \end{itemize}
    \begin{table}[h]
        \centering
        \begin{tabular}{c|c}
            Microsoft\textsuperscript{\textregistered}  Word & \LaTeX \\
            \hline
            文字处理工具 & 专业排版软件 \\
            容易上手,简单直观 & 容易上手 \\
            所见即所得 & 所见即所想,所想即所得 \\
            高级功能不易掌握 & 进阶难,但一般用不到 \\
            处理长文档需要丰富经验 & 和短文档处理基本无异 \\
            花费大量时间调格式 & 无需担心格式,专心作者内容 \\
            公式排版差强人意 & 尤其擅长公式排版 \\
            二进制格式,兼容性差 & 文本文件,易读、稳定 \\
            付费商业许可 & 自由免费使用 \\
        \end{tabular}
    \end{table}
\end{frame}

\begin{frame}{排版举例}
    \begin{exampleblock}{无编号公式} % 加 * 
        \begin{equation*}
            J(\theta) = \mathbb{E}_{\pi_\theta}[G_t] = \sum_{s\in\mathcal{S}} d^\pi (s)V^\pi(s)=\sum_{s\in\mathcal{S}} d^\pi(s)\sum_{a\in\mathcal{A}}\pi_\theta(a|s)Q^\pi(s,a)
        \end{equation*}
    \end{exampleblock}
    \begin{exampleblock}{多行多列公式\footnote{如果公式中有文字出现,请用 $\backslash$mathrm\{\} 或者 $\backslash$text\{\} 包含,不然就会变成 $clip$,在公式里看起来比 $\mathrm{clip}$ 丑非常多。}}
        % 使用 & 分隔
        \begin{align}
            Q_\mathrm{target}&=r+\gamma Q^\pi(s^\prime, \pi_\theta(s^\prime)+\epsilon)\\
            \epsilon&\sim\mathrm{clip}(\mathcal{N}(0, \sigma), -c, c)\nonumber
        \end{align}
    \end{exampleblock}
\end{frame}

\begin{frame}
    \begin{exampleblock}{编号多行公式}
        % Taken from Mathmode.tex
        \begin{multline}
            A=\lim_{n\rightarrow\infty}\Delta x\left(a^{2}+\left(a^{2}+2a\Delta x+\left(\Delta x\right)^{2}\right)\right.\label{eq:reset}\\
            +\left(a^{2}+2\cdot2a\Delta x+2^{2}\left(\Delta x\right)^{2}\right)\\
            +\left(a^{2}+2\cdot3a\Delta x+3^{2}\left(\Delta x\right)^{2}\right)\\
            +\ldots\\
            \left.+\left(a^{2}+2\cdot(n-1)a\Delta x+(n-1)^{2}\left(\Delta x\right)^{2}\right)\right)\\
            =\frac{1}{3}\left(b^{3}-a^{3}\right)
        \end{multline}
    \end{exampleblock}
\end{frame}


\begin{frame}[fragile]{\LaTeX{} 常用命令}
    \begin{exampleblock}{命令}
        \centering
        \footnotesize
        \begin{tabular}{llll}
            \cmd{chapter} & \cmd{section} & \cmd{subsection} & \cmd{paragraph} \\
            章 & 节 & 小节 & 带题头段落 \\\hline
            \cmd{centering} & \cmd{emph} & \cmd{verb} & \cmd{url} \\
            居中对齐 & 强调 & 原样输出 & 超链接 \\\hline
            \cmd{footnote} & \cmd{item} & \cmd{caption} & \cmd{includegraphics} \\
            脚注 & 列表条目 & 标题 & 插入图片 \\\hline
            \cmd{label} & \cmd{cite} & \cmd{ref} \\
            标号 & 引用参考文献 & 引用图表公式等\\\hline
        \end{tabular}
    \end{exampleblock}
    \begin{exampleblock}{环境}
        \centering
        \footnotesize
        \begin{tabular}{lll}
            \env{table} & \env{figure} & \env{equation}\\
            表格 & 图片 & 公式 \\\hline
            \env{itemize} & \env{enumerate} & \env{description}\\
            无编号列表 & 编号列表 & 描述 \\\hline
        \end{tabular}
    \end{exampleblock}
\end{frame}

\begin{frame}[fragile]{\LaTeX{} 环境命令举例}
    \begin{minipage}{0.5\linewidth}
\begin{lstlisting}[language=TeX]
\begin{itemize}
  \item A \item B
  \item C
  \begin{itemize}
    \item C-1
  \end{itemize}
\end{itemize}
\end{lstlisting}
    \end{minipage}\hspace{1cm}
    \begin{minipage}{0.3\linewidth}
        \begin{itemize}
            \item A
            \item B
            \item C
            \begin{itemize}
                \item C-1
            \end{itemize}
        \end{itemize}
    \end{minipage}
    \medskip
    \pause
    \begin{minipage}{0.5\linewidth}
\begin{lstlisting}[language=TeX]
\begin{enumerate}
  \item 巨佬 \item 大佬
  \item 萌新
  \begin{itemize}
    \item[n+e] 瑟瑟发抖
  \end{itemize}
\end{enumerate}
\end{lstlisting}
    \end{minipage}\hspace{1cm}
    \begin{minipage}{0.3\linewidth}
        \begin{enumerate}
            \item 巨佬
            \item 大佬
            \item 萌新
            \begin{itemize}
                \item[n+e] 瑟瑟发抖
            \end{itemize}
        \end{enumerate}
    \end{minipage}
\end{frame}

\begin{frame}[fragile]{\LaTeX{} 数学公式}
    \begin{columns}
        \begin{column}{.55\textwidth}
\begin{lstlisting}[language=TeX]
$V = \frac{4}{3}\pi r^3$

\[
  V = \frac{4}{3}\pi r^3
\]

\begin{equation}
  \label{eq:vsphere}
  V = \frac{4}{3}\pi r^3
\end{equation}
\end{lstlisting}
        \end{column}
        \begin{column}{.4\textwidth}
            $V = \frac{4}{3}\pi r^3$
            \[
                V = \frac{4}{3}\pi r^3
            \]
            \begin{equation}
                \label{eq:vsphere}
                V = \frac{4}{3}\pi r^3
            \end{equation}
        \end{column}
    \end{columns}
    \begin{itemize}
        \item 更多内容请看 \href{https://zh.wikipedia.org/wiki/Help:数学公式}{\color{purple}{这里}}
    \end{itemize}
\end{frame}

\begin{frame}[fragile]
    \begin{columns}
        \column{.6\textwidth}
\begin{lstlisting}[language=TeX]
    \begin{table}[htbp]
      \caption{编号与含义}
      \label{tab:number}
      \centering
      \begin{tabular}{cl}
        \toprule
        编号 & 含义 \\
        \midrule
        1 & 4.0 \\
        2 & 3.7 \\
        \bottomrule
      \end{tabular}
    \end{table}
    公式~(\ref{eq:vsphere}) 的
    编号与含义请参见
    表~\ref{tab:number}。
\end{lstlisting}
        \column{.4\textwidth}
        \begin{table}[htpb]
            \centering
            \caption{编号与含义}
            \label{tab:number}
            \begin{tabular}{cl}\toprule
                编号 & 含义 \\\midrule
                1 & 4.0\\
                2 & 3.7\\\bottomrule
            \end{tabular}
        \end{table}
        \normalsize 公式~(\ref{eq:vsphere})的编号与含义请参见表~\ref{tab:number}。
    \end{columns}
\end{frame}

\begin{frame}{作图}
    \begin{itemize}
        \item 矢量图 eps, ps, pdf
        \begin{itemize}
            \item METAPOST, pstricks, pgf $\ldots$
            \item Xfig, Dia, Visio, Inkscape $\ldots$
            \item Matlab / Excel 等保存为 pdf
        \end{itemize}
        \item 标量图 png, jpg, tiff $\ldots$
        \begin{itemize}
            \item 提高清晰度,避免发虚
            \item 应尽量避免使用
        \end{itemize}
    \end{itemize}
    \begin{figure}[htpb]
        \centering
        \includegraphics[width=0.2\linewidth]{pic/NJUPT_Logo.pdf
        }
        \caption{这个校徽就是矢量图,虽然看起来不像,但确实是矢量图格式}
    \end{figure}
\end{frame}

\section{计划进度}
\begin{frame}
    \begin{itemize}
        \item 一月:完成文献调研
        \item 二月:研究THU Beamer Theme的实现
        \item 三、四月:修改NJUPT Beamer主题
        \item 五月:论文撰写
    \end{itemize}
\end{frame}

\section{参考文献}

\begin{frame}[allowframebreaks]
    \bibliography{ref}
    \bibliographystyle{gbt}
\end{frame}

\begin{frame}
    \begin{center}
        {\Huge Thanks!}
    \end{center}
\end{frame}

\end{document}